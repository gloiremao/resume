\documentclass{article}
\usepackage{scimisc-cv}
\usepackage{hyperref}
\usepackage[utf8]{inputenc}

\title{Resume}
\author{Jerry Mao}
\date{May 2020}

\hypersetup{
    colorlinks=true,
    linkcolor=blue,    
    urlcolor=blue
    }

%% These are custom commands defined in scimisc-cv.sty
\cvname{Jerry Mao (Chen-Nien Mao)}
\cvpersonalinfo{
Senior Software Engineer at Yahoo! \cvinfosep 
+886 953701571 \cvinfosep
\href{mailto:chennien.mao@gmail.com}{chennien.mao@gmail.com} \cvinfosep
\href{https://www.linkedin.com/in/chennien-mao/}{LinkedIn:chennien-mao}
}

\begin{document}

% \maketitle %% This is LaTeX's default title constructed from \title,\author,\date

\makecvtitle %% This is a custom command constructing the CV title from \cvname, \cvpersonalinfo

\section{Experience}
\cvsubsection{Senior Software Engineer}[Oct. 2019 to present]
[Yahoo!][Taipei, Taiwan]
\begin{itemize}
\item \textbf{Read-time Local Search} (Back-end , Vespa, Java, k8s, Python) In this project, we provide the real-time location information data, e.g. Ubike, Bus schedule and parking information, to our map products. I have built a data pipeline to consume governments' open API and a search application to provide real-time data on our Yahoo mobile app.
\item \textbf{Yahoo Trending} (Back-end, PHP, Python, Hadoop) I have been working on the Yahoo Trending system since early 2020, I involved in the system migration from rhel6 to rhel7 and python2 to python3. Besides, I involve in trending revenue system for Taiwan and Hongkong markets and our goal is to generate the trending revenue keywords automatically.
\item \textbf{TV Series Search for Taiwan} (Back-end, Python, Vespa, Java) I am the lead engineer for Yahoo TV search which is the TV guide for user to search for the available TV sources from online streaming providers. In this project, I have built: 1) A web crawler which discovers the video content in possible online providers. 2) A data pipeline which processes the crawled data and transform them for search experience. 3) A search application based on \href{https://vespa.ai/}{Vespa} which powers the search scenarios on Yahoo Search ( \href{https://tw.search.yahoo.com/search?p=%E9%80%B2%E6%93%8A%E7%9A%84%E5%B7%A8%E4%BA%BA}{
Link}) and Yahoo App. 
\item \textbf{Handwriting System migration} (Back-end, AWS) I have been working on the AWS migration for handwriting system in early 2020. To reduce the maintanance cost and provide secure connection, I have built a deployment pipeline using Cloudformation to port the system on AWS and upgrade the API with mTLS support.


\end{itemize}

\cvsubsection{Software Engineer}[Aug. 2017 to Oct. 2019]
[Yahoo!][Taipei, Taiwan]
\begin{itemize}
\item \textbf{Yahoo Dictionary} (Front-end, Back-end, Vespa, Java, Python) I am the project owner of Yahoo Dictionary (\href{https://tw.dictionary.search.yahoo.com/}{link}). Since 2018, I have driven the dictionary interfaces redesign in Taiwan and HongKong markets which increase the daily active users by 12\%. Besides, I also maintained the dictionary back-end system, which powered all the search scenarios among Yahoo products.
\item \textbf{Yahoo Search} (Front-end) I first joined Yahoo as Search Front-end engineer, I have been involved in various Yahoo search interfaces development. In particular, my contributions include: 1) Implemented image ads at Yahoo search which increase revenue by 2\% in Taiwan markets 2) Improving the Local search with our new ajax framework which optimze the loading cost and increases 1\% user clicks and 3) Consolidate the carousel implementation which reduces the complexity and mainatance effort.
\end{itemize}

\cvsubsection{Research Assistant}[Nov. 2014 to Jun. 2017]
[Networking and Multimedia Systems Lab (NMSL), National Tsing Hua University][Hsinchu, Taiwan]
\begin{itemize}
\item I have been with NMSL since my senior year as a research assistant, and contributed to several projects. \begin{itemize}
\item \textbf{5G Cloud} (Advisors: Dr. Cheng-Hsin Hsu and Dr. Yeh-Chin Chung):  I studied and deployed an opensource project, Open Air Interface, which implements the Radio Access Networks in software. We virtualizes the Radio Access Networks in the cloud using the Docker containers and Kubernates.
\item \textbf{SDN in Enterprise Networks} (Advisors:  Dr. Cheng-Hsin Hsu): In this project, we leverage the MPLS concepts to design a low latency SDN algorithm for enterprise networks. The goal is to minimize the initialization delay and support load balance in the networks.
\end{itemize}
\end{itemize}

\cvsubsection{Research Intern}[Aug. 2015]
[Wireless and Mobile Communication Technology R\&D Center, TsingHua University][Beijing, China]
\begin{itemize}
\item I participated in the 5G project, which is a joint project between National Tsing Hua University, Hsinchu and TsingHua University, Beijing. I was invited by TsingHua
University to share the experience on 5G virtualization.
\end{itemize}

\section{Education}
\cvsubsection{M.S., Department of Computer Science, National Tsing Hua University}[Hsinchu, Taiwan]
[Networking and Multimedia Systems Lab][Sept 2015 to Jun 2017]

\begin{itemize}
\item Research Fields : Computer Networks, Radio Access Networks, Cloud, Software Defined
Networks
\item GPA 4.16/4.3
\end{itemize}

\cvsubsection{B.S., Department of Computer Science, National Tsing Hua University}[Hsinchu, Taiwan]
[Networking and Multimedia Systems Lab][Sept 2011 to Jun 2015]
\begin{itemize}
\item Selective Courses: Software Studio Lab (A+), Web Programming (A+), Introduction to Multimedia (A+), Introduction to Computer Networks (A+), Introduction to Communications (A+), Competitive Programming Training (A), Java Programming Language (A), Introduction to Programming (A)
\item GPA 3.77/4.3 (3.9 for Computer Science Courses)
\end{itemize}

\section{Publication}
\begin{itemize}
\item  \textbf{Chen-Nien Mao}, Mu-Han Huang, Satyajit Padhy, Shu-Ting Wang, Wu-Chun Chung, Yeh-Ching Chung,and Cheng-Hsin Hsu, "Minimizing Latency of Real-Time Container Cloud for Software Radio Access Networks.", in Proc. of IEEE 7th International Conference on Cloud Computing Technology and Science,
2015  
\item Mu-Han Huang ,Yu-Cing Luo and \textbf{Chen-Nien Mao}, Bing-Liang Chen and et al, "Performance evaluations of cloud radio access networks." in Proc. of 12th EAI International Conference on Heterogeneous Networking for Quality, Reliability, Security and Robustness, 2016
\end{itemize}

\section{Skills}
\begin{description}[widest=Langauges]
\item[Technical Skills:]
\textbf{Python:} 5 years experience,
\textbf{Java:} 2 years experience,
\textbf{C:} 3 years experience
\item[Languages:] \textbf{English:} Professional Proficiency. \textbf{Mandarin:} Native.
\end{description}


\end{document}
