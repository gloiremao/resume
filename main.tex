\documentclass{article}
\usepackage{scimisc-cv}
\usepackage{hyperref}
\usepackage[utf8]{inputenc}

\title{Resume}
\author{Jerry Mao}
\date{Oct. 2020}

\hypersetup{
    colorlinks=true,
    linkcolor=blue,    
    urlcolor=blue
    }

%% These are custom commands defined in scimisc-cv.sty
\cvname{Jerry Mao (Chen-Nien Mao)}
\cvpersonalinfo{
Senior Software Engineer at Yahoo! \cvinfosep 
+886 953701571 \cvinfosep
\href{mailto:chennien.mao@gmail.com}{chennien.mao@gmail.com} \cvinfosep
\href{https://www.linkedin.com/in/chennien-mao/}{LinkedIn:chennien-mao}
}

\begin{document}

% \maketitle %% This is LaTeX's default title constructed from \title,\author,\date

\makecvtitle %% This is a custom command constructing the CV title from \cvname, \cvpersonalinfo

\section{Education}
\cvsubsection{M.S., Department of Computer Science, National Tsing Hua University}[Hsinchu, Taiwan]
[Networking and Multimedia Systems Lab][Sept 2015 to Jun 2017]

\begin{itemize}
% \item Research Fields : Computer Networks, Radio Access Networks, Cloud, Software Defined Networks
\item GPA 4.16/4.3
\end{itemize}

\cvsubsection{B.S., Department of Computer Science, National Tsing Hua University}[Hsinchu, Taiwan]
[Computer Science][Sept 2011 to Jun 2015]
\begin{itemize}
% \item Selective Courses: Software Studio Lab (A+), Web Programming (A+), Introduction to Multimedia (A+), Introduction to Computer Networks (A+), Competitive Programming Training (A), Java Programming Language (A), Introduction to Programming (A)
\item GPA 3.77/4.3 (3.9 for Computer Science Courses)
\end{itemize}

\section{Experience}
\cvsubsection{Senior Software Engineer}[Oct. 2019 to present]
[Yahoo!][Taipei, Taiwan]
\begin{itemize}

\item \textbf{Yahoo TrendingNow} (Big Data, Python, PHP) Yahoo TrendingNow discovers the trending topics among Yahoo networks. Based on the user search queries and real-time news content, we analyze the data and provide a list of popular terms to allow users to discover more content on the hot topics. My contributions include 1) Redesign the API and introduce monitoring and alerting, 2) Migrate platform from rhel6 to rhel7 and python2 to python3, 3) Optimize trending commercial keywords in Taiwan and Hong Kong markets. After we launch the new approach, 73\% of the revenue gain is observed since we introduced a more real-time trending keywords process.

\item \textbf{TV Series Search for Taiwan} (Search Application, Python, Java) TV Series Search is the TV guide for the users to explore the available sources from online streaming providers.  In this project, I have built: 
1) Crawlers that discover the video content from online service providers. 
2) Data pipelines that process all the crawled data and build a relational data structure. 
3) A optimized search application provides well-designed search experiences (\href{https://tw.search.yahoo.com/search?p=%E9%80%B2%E6%93%8A%E7%9A%84%E5%B7%A8%E4%BA%BA}{link}) to our users. The metrics show that we cover more than 1\% of the search traffic and 25\% of users who interact with this product. The performance results indicate higher engagement among our search applications.

\item \textbf{Read-time Local Search} (Kubernetes, Search Application, Java, Python) In this project, we provide real-time location information data, e.g. public rental bikes, and parking information, to our map products. The data is streaming into our location search service through our well-designed real-time data pipeline and then presented on our search products.

% \item \textbf{Handwriting System migration} (Back-end, AWS) I have been working on the AWS migration for handwriting system in early 2020. To reduce the maintanance cost and provide secure connection, I have built a deployment pipeline using Cloudformation to port the system on AWS and upgrade the API with mTLS support. %


\end{itemize}

\cvsubsection{Software Engineer}[Aug. 2017 to Oct. 2019]
[Yahoo!][Taipei, Taiwan]
\begin{itemize}
\item \textbf{Yahoo Dictionary} (Front-end, Search Application, Java, Python) I am the project owner of Yahoo Dictionary (\href{https://tw.dictionary.search.yahoo.com/}{link}). I have driven the dictionary interface redesign in Taiwan and Hong Kong markets that increase the daily active users by 12\%. Besides, I also maintained the dictionary back-end system, which powered all the search scenarios among Yahoo products.
\item \textbf{Yahoo Search} (Front-end, HTML, CSS, Javascript) I first joined Yahoo as a front-end engineer, I have been involved in various Yahoo search interfaces development. In particular, my highlighted contributions include: 1) Implemented image ads at Yahoo search which increase revenue by 2\% in Taiwan 2) Improving the Local search with Ajax framework which optimize the loading cost and increases 1\% user click. % and 3) Consolidate the carousel implementation which reduces the complexity and maintenance effort.
\end{itemize}

\cvsubsection{Research Assistant}[Nov. 2014 to Jun. 2017]
[Networking and Multimedia Systems Lab (NMSL), National Tsing Hua University][Hsinchu, Taiwan]
\begin{itemize}
\item I have been with NMSL since my senior year as a research assistant, and contributed to several projects. \begin{itemize}
% \item \textbf{5G Cloud} (Advisors: Dr. Cheng-Hsin Hsu and Dr. Yeh-Chin Chung):  I studied and deployed an opensource project, Open Air Interface, which implements the Radio Access Networks in software. We virtualizes the Radio Access Networks in the cloud using the Docker containers and Kubernates.
\item \textbf{SDN in Enterprise Networks} (Advisors:  Dr. Cheng-Hsin Hsu): In this project, we leverage the MPLS concepts to design a low latency SDN algorithm for enterprise networks. The goal is to minimize the initialization delay and support load balance in the networks.
\end{itemize}
\end{itemize}

%\cvsubsection{Research Intern}[Aug. 2015]
%[Wireless and Mobile Communication Technology R\&D Center, TsingHua University][Beijing, China]
%\begin{itemize}
%\item I participated in the 5G project, which is a joint project between National Tsing Hua University, Hsinchu and TsingHua University, Beijing. I was invited by TsingHua
%University to share the experience on 5G virtualization.
%\end{itemize}

\section{Publication}
\begin{itemize}
\item  \textbf{Chen-Nien Mao}, Mu-Han Huang, Satyajit Padhy, et al., "Minimizing Latency of Real-Time Container Cloud for Software Radio Access Networks.", in Proc. of IEEE 7th International Conference on Cloud Computing Technology and Science,
2015  
\item Mu-Han Huang ,Yu-Cing Luo and \textbf{Chen-Nien Mao}, et al., "Performance evaluations of cloud radio access networks." in Proc. of 12th EAI International Conference on Heterogeneous Networking for Quality, Reliability, Security and Robustness, 2016
\end{itemize}

\section{Skills}
\begin{description}[widest=Langauges]
\item[Programming Skills:]
\textbf{Python:} 5 years experience,
\textbf{Java:} 2 years experience,
\textbf{C:} 3 years experience
\item[Technical Skills:]
\textbf{AWS:} 1 year experience
\textbf{Kubernetes:} 1 year experience
\item[Languages:] 
\textbf{English:} Professional Proficiency.
\textbf{Mandarin:} Native.
\end{description}


\end{document}
